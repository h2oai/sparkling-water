\section{Productionizing MOJOs from H2O-3}

\subsection{Loading the H2O-3 MOJOs}

When training algorithm using Sparkling Water API, Sparkling Water always produces \texttt{H2OMOJOModel}. It is however also possible
to import existing MOJO into Sparkling Water ecosystem from H2O-3. After importing the H2O-3 MOJO the API is unified for the
loaded MOJO and the one created in Sparkling Water, for example, using \texttt{H2OXGBoost}.

H2O MOJOs can be imported to Sparkling Water from all data sources supported by Apache Spark such as local file, S3 or HDFS and the
semantics of the import is the same as in the Spark API.

When creating a MOJO specified by a relative path and HDFS is enabled, the method attempts to load
the MOJO from the HDFS home directory of the current user. In case we are not running on HDFS-enabled system, we create
the mojo from a current working directory.

\begin{itemize}
    \item \textbf{Scala} \begin{lstlisting}[style=Scala]
import ai.h2o.sparkling.ml.models._
val model = H2OMOJOModel.createFromMojo("prostate_mojo.zip")
    \end{lstlisting}
    \item \textbf{Python} \begin{lstlisting}[style=Python]
from pysparkling.ml import *
model = H2OMOJOModel.createFromMojo("prostate_mojo.zip")
    \end{lstlisting}
    \item \textbf{R} \begin{lstlisting}[style=R]
library(rsparkling)
sc <- spark_connect(master = "local")
model <- H2OMOJOModel.createFromMojo("prostate_mojo.zip")
    \end{lstlisting}
\end{itemize}


Absolute local path can also be used. To create a MOJO model from a locally available MOJO, call:

\begin{itemize}
    \item \textbf{Scala} \begin{lstlisting}[style=Scala]
import ai.h2o.sparkling.ml.models._
val model = H2OMOJOModel.createFromMojo("/Users/peter/prostate_mojo.zip")
    \end{lstlisting}
    \item \textbf{Python} \begin{lstlisting}[style=Python]
from pysparkling.ml import *
model = H2OMOJOModel.createFromMojo("/Users/peter/prostate_mojo.zip")
    \end{lstlisting}
    \item \textbf{R} \begin{lstlisting}[style=R]
library(rsparkling)
sc <- spark_connect(master = "local")
model <- H2OMOJOModel.createFromMojo("/Users/peter/prostate_mojo.zip")
    \end{lstlisting}
\end{itemize}


Absolute paths on Hadoop can also be used. To create a MOJO model from a MOJO stored on HDFS, call:

\begin{itemize}
    \item \textbf{Scala} \begin{lstlisting}[style=Scala]
import ai.h2o.sparkling.ml.models._
val model = H2OMOJOModel.createFromMojo("/user/peter/prostate_mojo.zip")
    \end{lstlisting}
    \item \textbf{Python} \begin{lstlisting}[style=Python]
from pysparkling.ml import *
model = H2OMOJOModel.createFromMojo("/user/peter/prostate_mojo.zip")
    \end{lstlisting}
    \item \textbf{R} \begin{lstlisting}[style=R]
library(rsparkling)
sc <- spark_connect(master = "local")
model <- H2OMOJOModel.createFromMojo("/user/peter/prostate_mojo.zip")
    \end{lstlisting}
\end{itemize}


The call loads the mojo file from the following location\\ \texttt{hdfs://{server}:{port}/user/peter/prostate\_mojo.zip},
where \texttt{server} and \texttt{port} are automatically filled in by Spark.


We can also manually specify the type of data source we need to use, in that case, we need to provide the schema:

\begin{itemize}
    \item \textbf{Scala} \begin{lstlisting}[style=Scala]
import ai.h2o.sparkling.ml.models._
// HDFS
val modelHDFS = H2OMOJOModel.createFromMojo("hdfs:///user/peter/prostate_mojo.zip")
// Local file
val modelLocal = H2OMOJOModel.createFromMojo("file:///Users/peter/prostate_mojo.zip")
    \end{lstlisting}
    \item \textbf{Python} \begin{lstlisting}[style=Python]
from pysparkling.ml import *
# HDFS
modelHDFS = H2OMOJOModel.createFromMojo("hdfs:///user/peter/prostate_mojo.zip")
# Local file
modelLocal = H2OMOJOModel.createFromMojo("file:///Users/peter/prostate_mojo.zip")
    \end{lstlisting}
    \item \textbf{R} \begin{lstlisting}[style=R]
library(rsparkling)
sc <- spark_connect(master = "local")
# HDFS
modelHDFS <- H2OMOJOModel.createFromMojo("hdfs:///user/peter/prostate_mojo.zip")
# Local file
modelLocal <- H2OMOJOModel.createFromMojo("file:///Users/peter/prostate_mojo.zip")
    \end{lstlisting}
\end{itemize}


The loaded model is an immutable instance, so it's not possible to change the configuration of the model during its existence.
On the other hand, the model can be configured during its creation via \texttt{H2OMOJOSettings}:

\begin{itemize}
    \item \textbf{Scala} \begin{lstlisting}[style=Scala]
import ai.h2o.sparkling.ml.models._
val settings = H2OMOJOSettings(convertUnknownCategoricalLevelsToNa = true, convertInvalidNumbersToNa = true)
val model = H2OMOJOModel.createFromMojo("prostate_mojo.zip", settings)
    \end{lstlisting}
    \item \textbf{Python} \begin{lstlisting}[style=Python]
from pysparkling.ml import *
settings = H2OMOJOSettings(convertUnknownCategoricalLevelsToNa = True, convertInvalidNumbersToNa = True)
model = H2OMOJOModel.createFromMojo("prostate_mojo.zip", settings)
    \end{lstlisting}
    \item \textbf{R} \begin{lstlisting}[style=R]
library(rsparkling)
sc <- spark_connect(master = "local")
settings <- H2OMOJOSettings(convertUnknownCategoricalLevelsToNa = TRUE, convertInvalidNumbersToNa = TRUE)
model <- H2OMOJOModel.createFromMojo("prostate_mojo.zip", settings)
    \end{lstlisting}
\end{itemize}

To score the dataset using the loaded mojo, call:

\begin{itemize}
    \item \textbf{Scala} \begin{lstlisting}[style=Scala]
model.transform(dataset)
    \end{lstlisting}
    \item \textbf{Python} \begin{lstlisting}[style=Python]
model.transform(dataset)
    \end{lstlisting}
    \item \textbf{R} \begin{lstlisting}[style=R]
model$transform(dataset)
    \end{lstlisting}
\end{itemize}


In Scala, the \texttt{createFromMojo} method returns a mojo model instance casted as a base class \texttt{H2OMOJOModel}. This class holds
only properties that are shared accross all MOJO model types from the following type hierarchy:

\begin{itemize}
    \item \texttt{H2OMOJOModel}
    \item \texttt{H2OUnsupervisedMOJOModel}
    \item \texttt{H2OSupervisedMOJOModel}
    \item \texttt{H2OTreeBasedSupervisedMOJOModel}
\end{itemize}

If a Scala user wants to get a property specific for a given MOJO model type, they must utilize casting or
call the \texttt{createFromMojo} method on the specific MOJO model type.

\begin{lstlisting}[style=Scala]
import ai.h2o.sparkling.ml.models._
val specificModel = H2OTreeBasedSupervisedMOJOModel.createFromMojo("prostate_mojo.zip")
println(s"Ntrees: ${specificModel.getNTrees()}") // Relevant only to GBM, DRF and XGBoost
\end{lstlisting}

\subsection{Exporting the loaded MOJO model using Sparkling Water}

To export the MOJO model, call \texttt{model.write.save(path)}. In case of Hadoop enabled system, the command by default
uses HDFS.

\subsection{Importing the previously exported MOJO model from Sparkling Water}

To import the Sparkling Water MOJO model, call\\ \texttt{H2OMOJOModel.read.load(path)}. In case of Hadoop enabled system, the command by default
uses HDFS.

\subsection{Accessing additional prediction details}

After computing predictions, the \texttt{prediction} column contains in case of classification problem the predicted label
and in case regression problem the predicted number. If we need to access more details for each prediction, , see the content
of a detailed prediction column. By default, the column is named named \texttt{detailed\_prediction} . It could contain, for example,
predicted probabilities for each predicted label in case of classification problem, shapley values and other information.

\subsection{Customizing the MOJO Settings}

We can configure the output and format of predictions via the \texttt{H2OMOJOSettings}. The available options are:

\begin{itemize}
    \item \texttt{predictionCol} - Specifies the name of the generated prediction column. Default value is \texttt{prediction}.
    \item \texttt{detailedPredictionCol} - Specifies the name of the generated detailed prediction column. The detailed prediction column, if enabled, contains additional details, such as probabilities, shapley values etc. The default value is \texttt{detailed\_prediction}.
    \item \texttt{convertUnknownCategoricalLevelsToNa} - Enables or disables conversion of unseen categoricals to NAs. By default, it is disabled.
    \item \texttt{convertInvalidNumbersToNa} - Enables or disables conversion of invalid numbers to NAs. By default, it is disabled.
    \item \texttt{withContributions} - Enables or disables computing Shapley values. Shapley values are generated as a sub-column for the detailed prediction column.
    \item \texttt{withLeafNodeAssignments} - When enabled, a user can obtain the leaf node assignments after the model training has finished. By default, it is disabled.
    \item \texttt{withStageResults} - When enabled, a user can obtain the stage results for tree-based models. By default, it is disabled and also it's not supported by XGBoost although it's a tree-based algorithm.
\end{itemize}

\subsection{Methods available on MOJO Model}

\subsubsection{Obtaining Domain Values}

To obtain domain values of the trained model, we can run \texttt{getDomainValues()} on the model. This call
returns a mapping from a column name to its domain in a form of array.

\subsubsection{Obtaining Model Category}

The method \texttt{getModelCategory} can be used to get the model category (such as \texttt{binomial}, \texttt{multinomial} etc).

\subsubsection{Obtaining Training Params}

The method \texttt{getTrainingParams} can be used to get map containing all training parameters used in the H2O. It is a map
from parameter name to the value. The parameters name use the H2O's naming structure.


\subsubsection{Obtaining Metrics}

There are several methods to obtain metrics from the MOJO model. All return a map from metric name to its double value.

\begin{itemize}
    \item \texttt{getTrainingMetrics} - obtain training metrics
    \item \texttt{getValidationMetrics} - obtain validation metrics
    \item \texttt{getCrossValidationMetrics} - obtain cross validation metrics
\end{itemize}


We also have method \texttt{getCurrentMetrics} which gets one of the metrics above based on the following algorithm:

If cross validation was used, ie, \texttt{setNfolds} was called and value was higher than zero, this method returns cross validation
metrics. If cross validation was not used, but validation frame was used, the method returns validation metrics. Validation
frame is used if \texttt{setSplitRatio} was called with value lower than one. If neither cross validation or validation frame
was used, this method returns the training metrics.

\subsubsection{Obtaining Leaf Node Assignments}

To obtain the leaf node assignments, please first make sure to set\\ \texttt{withLeafNodeAssignments}
on your MOJO settings object. The leaf node assignments are now stored
in the \texttt{\${detailedPredictionCol}.leafNodeAssignments} column on the dataset obtained from the prediction.
Please replace\\ \texttt{\${detailedPredictionCol}} with the actual value of your detailed prediction col. By default,
it is \texttt{detailed\_prediction}.

\subsubsection{Obtaining Stage Probabilities}

To obtain the stage results, please first make sure to set \texttt{withStageResults} and
\texttt{withDetailedPredictionCol} to true on your MOJO settings object. The stage results for regression and
anomaly detection problems
are stored in the \texttt{\${detailedPredictionCol}.stageResults} on the dataset obtained from the prediction. The
stage results for classification(binomial, multinomial) problems are stored under \\
\texttt{\${detailedPredictionCol}.stageProbabilities}. Please replace \\ \texttt{\${detailedPredictionCol}}
with the actual value of your detailed prediction col. By default, it is \texttt{detailed\_prediction}.

The stage results are an array of values, where a value at the position *t* is the prediction/probability combined
from contributions of trees *T1, T2, ..., Tt*. For *t* equal to a number of model trees, the value is the same as the
final prediction/probability. The stage results (probabilities) for the classification problem
are represented by a list of columns, where one column contains stage probabilities for a given prediction class.

\section{Productionizing MOJOs from Driverless AI}

MOJO scoring pipeline artifacts, created in Driverless AI, can be used in Spark to carry out predictions in parallel
using the Sparkling Water API. This section shows how to load and run predictions on the MOJO scoring pipeline in
Sparkling Water.

Note: Sparkling Water is backward compatible with MOJO versions produced by different Driverless AI versions.

One advantage of scoring the MOJO artifacts is that \texttt{H2OContext} does not have to be created if we only want to
run predictions on MOJOs using Spark. This is because the scoring is independent of the H2O run-time. It is also
important to mention that the format of prediction on MOJOs from Driverless AI differs from predictions on H2O-3 MOJOs.
The format of Driverless AI prediction is explained bellow.

\subsection{Requirements}

In order to use the MOJO scoring pipeline, Driverless AI license has to be passed to Spark.
This can be achieved via \texttt{--jars} argument of the Spark launcher scripts.

Note: In Local Spark mode, please use \texttt{--driver-class-path} to specify the path to the license file.

We also need Sparkling Water distribution which can be obtained from \url{https://www.h2o.ai/download/}.
After we downloaded the Sparkling Water distribution, extract it, and go to the extracted directory.

\subsection{Loading and Score the MOJO}

First, start the environment for the desired language with Driverless AI license. There are two variants. We can use
Sparkling Water prepared scripts which put required dependencies on the Spark classpath or we can use Spark directly
and add the dependencies manually.

In case of Scala, run:
\begin{lstlisting}[style=Bash]
./bin/spark-shell --jars license.sig,jars/sparkling-water-assembly_2.12-3.30.1.1-1-3.0-all.jar
\end{lstlisting}

or

\begin{lstlisting}[style=Bash]
./bin/sparkling-shell --jars license.sig
\end{lstlisting}

In case of Python, run:

\begin{lstlisting}[style=Bash]
./bin/pyspark --jars license.sig --py-files py/build/dist/h2o_pysparkling_3.0-3.30.1.1-1-3.0.zip
\end{lstlisting}

or

\begin{lstlisting}[style=Bash]
./bin/pysparkling --jars license.sig
\end{lstlisting}

In case of R, run in R console or RStudio:

\begin{lstlisting}[style=R]
library(sparklyr)
library(rsparkling)
config <- spark_config()
config$sparklyr.jars.default <- "license.sig"
spark_connect(master = "local", version = "3.0.0", config = config)
\end{lstlisting}


At this point, we should have Spark interactive terminal where we can carry out predictions.
For productionalizing the scoring process, we can use the same configuration,
except instead of using Spark shell, we would submit the application using \texttt{./bin/spark-submit}.

Now Load the MOJO as:

\begin{itemize}
    \item \textbf{Scala} \begin{lstlisting}[style=Scala]
import ai.h2o.sparkling.ml.models.H2OMOJOPipelineModel
val settings = H2OMOJOSettings(predictionCol = "fruit_type", convertUnknownCategoricalLevelsToNa = true)
val mojo = H2OMOJOPipelineModel.createFromMojo("file:///path/to/the/pipeline_mojo.zip", settings)
    \end{lstlisting}
    \item \textbf{Python} \begin{lstlisting}[style=Python]
from pysparkling.ml import H2OMOJOPipelineModel
settings = H2OMOJOSettings(predictionCol = "fruit_type", convertUnknownCategoricalLevelsToNa = True)
mojo = H2OMOJOPipelineModel.createFromMojo("file:///path/to/the/pipeline_mojo.zip", settings)
    \end{lstlisting}
    \item \textbf{R} \begin{lstlisting}[style=R]
library(rsparkling)
settings <- H2OMOJOSettings(predictionCol = "fruit_type", convertUnknownCategoricalLevelsToNa = TRUE)
mojo <- H2OMOJOPipelineModel.createFromMojo("file:///path/to/the/pipeline_mojo.zip", settings)
    \end{lstlisting}
\end{itemize}

In the examples above \texttt{settings} is an optional argument. If it's not specified, the default values are used.

Prepare the dataset to score on:

\begin{itemize}
    \item \textbf{Scala} \begin{lstlisting}[style=Scala]
val dataFrame = spark.read.option("header", "true").option("inferSchema", "true").csv("file:///path/to/the/data.csv")
    \end{lstlisting}
    \item \textbf{Python} \begin{lstlisting}[style=Python]
dataFrame = spark.read.option("header", "true").option("inferSchema", "true").csv("file:///path/to/the/data.csv")
    \end{lstlisting}
    \item \textbf{R} \begin{lstlisting}[style=R]
dataFrame <- spark_read_csv(sc, name = "table_name", path = "file:///path/to/the/data.csv", header = TRUE)
    \end{lstlisting}
\end{itemize}

And finally, score the mojo on the loaded dataset:

\begin{itemize}
    \item \textbf{Scala} \begin{lstlisting}[style=Scala]
val predictions = mojo.transform(dataFrame)
    \end{lstlisting}
    \item \textbf{Python} \begin{lstlisting}[style=Python]
predictions = mojo.transform(dataFrame)
    \end{lstlisting}
    \item \textbf{R} \begin{lstlisting}[style=R]
predictions <- mojo$$transform(dataFrame)
    \end{lstlisting}
\end{itemize}

We can select the predictions as:

\begin{itemize}
    \item \textbf{Scala} \begin{lstlisting}[style=Scala]
predictions.select("prediction")
    \end{lstlisting}
    \item \textbf{Python} \begin{lstlisting}[style=Python]
predictions.select("prediction")
    \end{lstlisting}
    \item \textbf{R} \begin{lstlisting}[style=R]
predictions <- select(dataFrame, "prediction")
    \end{lstlisting}
\end{itemize}


The output data frame contains all the original columns plus the prediction column which is by default named
\texttt{prediction}. The prediction column contains all the prediction detail. Its name can be
modified via the \texttt{H2OMOJOSettings} object.

\subsection{Predictions Format}

When the option \texttt{namedMojoOutputColumns} is enabled on\\ \texttt{H2OMOJOSettings}, the \texttt{predictionCol} contains sub-columns with
names corresponding to the columns Driverless AI identified as output columns. For example, if Driverless API MOJO
pipeline contains one output column \texttt{AGE} ( for example regression problem), the prediction column contains another sub-column
named \texttt{AGE}. If The MOJO pipeline contains multiple output columns, such as \texttt{VALID.0} and \texttt{VALID.1} (for example classification problems),
the prediction column contains two sub-columns with the aforementined names.

If this option is disabled, the \texttt{predictionCol} contains the array of predictions without
the column names. For example, if Driverless API MOJO pipeline contains one output column \texttt{AGE} ( for example regression problem),
the prediction column contains array of size 1 with the predicted value.
If The MOJO pipeline contains multiple output columns, such as \texttt{VALID.0} and \texttt{VALID.1} (for example classification problems),
the prediction column contains array of size 2 containing predicted probabilities for each class.

By default, this option is enabled.

\subsection{Customizing the MOJO Settings}

We can configure the output and format of predictions via the H2OMOJOSettings. The available options are

\begin{itemize}
    \item \texttt{predictionCol} - Specifies the name of the generated prediction column. The default value is \texttt{prediction}.
    \item \texttt{convertUnknownCategoricalLevelsToNa} - Enables or disables conversion of unseen categoricals to NAs. By default, it is disabled.
    \item \texttt{convertInvalidNumbersToNa} - Enables or disables conversion of invalid numbers to NAs. By default, it is disabled.
    \item \texttt{namedMojoOutputColumns} - Enables or disables named output columns. By default, it is enabled.
\end{itemize}


\subsection{Troubleshooting}

If you see the following exception during loading the MOJO pipeline:\\
java.io.IOException: MOJO doesn't contain \texttt{resource mojo/pipeline.pb}, then it means you are adding
incompatible mojo-runtime.jar on your classpath. It is not required and also not suggested
to put the JAR on the classpath as Sparkling Water already bundles the correct dependencies.
